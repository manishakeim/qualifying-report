%! Author = mkeim
%! Date = 7/11/24

% Document
\section{Abstract} \label{sec:abstract}
    In the digital age, online information seekers frequently encounter search results laden with misinformation, biased narratives, and conspiracy theories.
    This exposure can potentially lead users to accept and propagate false information.
    However, this phenomenon is not universal across all search queries.
    Certain searches, known as \textit{data voids} are particularly susceptible to these issues.
    \\
    \textit{Data voids} occur when searching for terms that yield limited, non-existent, or highly problematic relevant information.
    Unlike common searches that produce abundant data, queries falling into data voids may return no results or present irrelevant and often inaccurate information.
    Data voids can be categorized into two main types: newly coined terms with no established information base, and existing terms whose meanings have evolved over time.
    This report focuses on the latter, examining data voids where context drift occurs.
    The following papers has been selected to explore this phenomenon.
    \begin{itemize}
        \item Statistically Significant Detection of Linguistic Change. ACM WWW 2014 \cite{kulkarni2014statisticallysignificantdetectionlinguistic}
        \item Diachronic Word Embeddings Reveal Statistical Laws of Semantic Change. ACL 2016 \cite{hamilton-etal-2016-diachronic}
        \item The Pandemic in Words: Tracking Fast Semantic Changes via a Large-Scale Word Association Task. \cite{10.1162/opmi_a_00081}
    \end{itemize}
%! Author = mkeim
%! Date = 7/11/24

\section{Introduction} \label{sec:introduction}
Language is a dynamic, living entity that continually adapts to reflect the changing world around us.
With the evolving language, words take on new meanings through technological advancements, cultural shifts, and political movements.

Particularly in political contexts, certain words or phrases can experience rapid shifts in meaning or become ambiguous.
These are known as `data voids' - situations where the meaning of a term has become unstable or contested, often due to deliberate efforts by political actors to redefine it.

Unlike more general language evolution, where new terms emerge to describe novel concepts, data voids in politics involve the strategic manipulation of existing vocabulary.
Political actors may repurpose words or introduce new phrases to frame debates in their favor, creating confusion and a lack of shared understanding.

This phenomenon highlights how language can be weaponized as a political tool.
As terms take on drastically different meanings for different groups, it can become challenging to maintain clear, authoritative information about their usage.
The resulting `data voids' can then be exploited to spread misinformation or propaganda.

My research goal is to find data voids in political context automatically and systematically study how word shifts their meaning over time.
To achieve this, we have selected following research papers as a foundation for our research.

\begin{itemize}
    \item Statistically Significant Detection of Linguistic Change.
    ACM WWW 2015 ~\cite{kulkarni2014statisticallysignificantdetectionlinguistic}
    \item Diachronic Word Embeddings Reveal Statistical Laws of Semantic Change.
    ACL 2016 ~\cite{hamilton-etal-2016-diachronic}
\end{itemize}

The concept of context drift, as presented in these papers, refers to the change in the contextual usage of words over time.
This can be particularly useful for detecting and analyzing data voids in political contexts.

Finding data voids is significantlly important because
political events or campaigns can cause sudden changes in how certain terms are used or interpreted.
Political actors may deliberately redefine terms or introduce new phrases to frame debates in their favor.
Words or phrases can take on drastically different meanings in different political contexts or for different groups.
Some terms may develop coded political meanings that aren't immediately apparent to all audiences.
The lack of clear, authoritative information about a term's new political meaning can be exploited to spread misinformation or propaganda.

These elements can rapidly change the basic meaning (denotation) and the feelings or ideas they evoke (connotation) of words, making them challenging to track and analyze.

\para{Example.}
The term \emph{crisis actor} has undergone a significant shift in meaning,
exemplifying the phenomenon of context drift we've been discussing.

Originally this term was used in the entertainment and training industries,
which refers to individuals hired to portray victims or other roles in simulated emergency scenarios.
These simulations were used for training first responders, military personnel, or for other educational purposes.

However, in recent years, the term has been full of misinformation which refers to genuine victims of violent events (mass shooting) in an attempt to prove that an event was staged or did not happen.
This term got prominent during Sandy Hook Elementary School shooting (2012), Los Angeles International Airport (LAX) shooting (2013), Pulse Nightclub shooting in Orlando (2016), Las Vegas shooting (2017).

This example clearly demonstrates how a term with a specific, legitimate meaning can be repurposed to spread misinformation and conspiracy theories.
The shift creates a potential data void where searches for \emph{crisis actor} might lead to a mix of factual information about emergency training and harmful conspiracy theories,
making it difficult for users to distinguish accurate information.
%! Author = mkeim
%! Date = 7/11/24

\section{Introduction} \label{sec:introduction}
In the digital age, online information seekers frequently encounter \emph{search results} laden with \emph{misinformation}, \emph{biased narratives}, and \emph{conspiracy theories}.
This exposure can potentially lead users to accept and propagate false information.
However, this phenomenon is not universal across all search queries.
Certain searches, known as \textit{data voids} are particularly susceptible to these issues.
\\
\textit{Data voids} occur when searching for terms that yield limited, non-existent, or highly problematic relevant information.
Unlike common searches that produce abundant data, queries falling into data voids may return no results or present irrelevant and often inaccurate information.
Data voids can be mainly categorized into two main types:
\begin{itemize}
    \item newly coined terms with no established information base, and
    \item existing terms whose meanings have evolved over time.
\end{itemize}
This report focuses on the latter, examining data voids where context drift occurs.
The following papers has been selected to explore this phenomenon.
\begin{itemize}
    \item Statistically Significant Detection of Linguistic Change.
    ACM WWW 2014 ~\cite{kulkarni2014statisticallysignificantdetectionlinguistic}
    \item Diachronic Word Embeddings Reveal Statistical Laws of Semantic Change.
    ACL 2016 ~\cite{hamilton-etal-2016-diachronic}
\end{itemize}

\subsection{What is a Data Void?}\label{subsec:what-is-a-data-void?}
Data voids, as first introduced by Golebiewski and Boyd \cite{unknown},
are identified while searching terms for which relevant information is either limited, non-existent, or deeply problematic.
\\
For example, unlike a typical web search for `basketball' that yields a vast amount of data,
data voids present a problematic scenario such as searches on these terms may return no results or irrelevant, often inaccurate information.
This situation could prompt users to pursue information, click on misleading search outcomes, and be exposed to false or misleading content,
ultimately shaping their perceptions and beliefs with misinformation.

Data Voids often originate in fringe communities.
When vulnerable users search for information using this term, they may encounter results that are misrepresentations of the topic, as these fringe communities have already shaped the narrative.
Mainstream media, particularly those with far-left or far-right biases, may pick up and amplify these narratives, contributing to the misinformation cycle.

\subsection{Example of a data void?(Migrant Caravan)}
Let's understand data void through an example of a term - \emph{migrant caravan}
The term ‘migrant caravan’ originally described a group of migrants traveling together for safety and mutual support.
However, its usage, particularly in political discussions, has taken on a more negative tone.
Today, it is commonly associated with large groups of migrants, often from Central America, journeying through Mexico to reach the United States.


\subsection{Lifecycle of data void (from long tail to spike)}
\subsection{Types of Data Void}
\subsection{Technical Challenge}
analyzing data void involves context drift over historical time. (How context drift happens with data void.)

%! Author = mkeim
%! Date = 7/11/24

\section{Introduction} \label{sec:introduction}
Language is a dynamic, living entity that continually adapts to reflect the changing world around us.
As language evolves, words shift in meaning, acquire new connotations, or are repurposed for entirely new contexts.
These changes occur as society, technology, and culture change.
In the digital age, search engines and online platforms play a crucial role in shaping access to knowledge and information.
When words change meaning, there is often a delay before search engines and online platforms fully reflect these new meanings.
This creates a \emph{data void} — an absence of relevant, up-to-date information based on the current understanding of the word.
This void represents a gap between evolving language and existing online content.

\para{Research goal.}
Understanding how words evolve is crucial because it allows us to anticipate where data voids might occur.
By tracking linguistic changes, we aim to predict when a word might create a gap in digital information which could lead to a potential data void.
To achieve this, we have selected the following research papers as a foundation for our study.

\begin{itemize}
    \item Statistically Significant Detection of Linguistic Change.
    ACM WWW 2015 ~\cite{kulkarni2014statisticallysignificantdetectionlinguistic} (\Cref{sec:paper_kulkarni})
    \item Diachronic Word Embeddings Reveal Statistical Laws of Semantic Change.
    ACL 2016 ~\cite{hamilton-etal-2016-diachronic} (\Cref{sec:paper_hamilton})
\end{itemize}

These papers present the concept of semantic change, which refers to the evolution of a word’s contextual usage over time.
This approach is particularly useful for analyzing how words evolve and ensuring that, as meanings shift, data voids are promptly recognized and addressed.

\para{Finding data voids is significantly important.}
Data voids have real-world implications, particularly in socio-political and sociotechnical domains.
When important terms, especially those tied to politics, social issues, or technology, gain new meanings,
people can exploit data voids to shape public perception.

Data voids, left unaddressed, can distort debates, influence elections, and shape collective beliefs, often in misleading or harmful ways.
Once a data void is identified, it’s critical to fill that gap with accurate, authoritative information.
Otherwise, bad actors can take advantage of the absence of reliable data to promote their own agendas.
By understanding linguistic changes, researchers and technology platforms can work together to prevent harmful or misleading content from filling these voids.

\para{Example.}
The term \emph{crisis actor} has significantly shifted in meaning,
exemplifying the phenomenon of semantic change we've been discussing.

Historically, `crisis actor' referred to actors employed in emergency preparedness training exercises.
These actors simulated victims or other roles in staged disaster drills to help first responders practice their response to crises like natural disasters or terrorist attacks.
It was a technical, niche term with a specific and legitimate use.

However, in recent years, it is used to describe the false idea that victims of real-world tragedies are actually actors.
For example, in mass shootings, propaganda and misinformation campaigns have referred to victims as “crisis actors,” falsely implying that the events were staged or fabricated.

This example clearly demonstrates how a data void occurs when people co-opt a word’s meaning and cause it to evolve rapidly,
leaving a gap in reliable, accurate content.
This allows misinformation to flourish, having a profound impact on both political discourse and social trust.\\

This report introduces methodologies to identify shifts in words and this will be useful in our research to proactively identify data voids based on two key papers.
To contextualize our work, we provide a thorough literature review examining the evolution of word meaning (\Cref{sec:relatedwork}).
Subsequently, Section \Cref{sec:paper_kulkarni} delves into the first paper, focusing on significant linguistic change detection,
while Section \Cref{sec:paper_hamilton} explores the second paper on diachronic word embeddings.
















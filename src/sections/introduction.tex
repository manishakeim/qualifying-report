%! Author = mkeim
%! Date = 7/11/24

\section{Introduction} \label{sec:introduction}
Language is a dynamic, living entity that continually adapts to reflect the changing world around us.
With the evolving language, words take on new meanings or connotations with technological advancements,
social and cultural shifts, generational differences, globalization, political and social movements, etc.
The particular words when get emerged in political context are known as data voids.
Since, they have the behavior of changing meaning over time.
There can be "voids" in the meaning of words or concepts, especially in specific contexts like politics.
When a word's meaning shifts rapidly or becomes ambiguous, it can create a kind of "data void" for analysis or understanding.

Data voids in political contexts often occur when terms or phrases take on new or altered meanings,
particularly when these changes are driven by political actors or movements.

This type of data void is especially significant because:

Rapid meaning shifts: Political events or campaigns can cause sudden changes in how certain terms are used or interpreted.
Intentional manipulation: Political actors may deliberately redefine terms or introduce new phrases to frame debates in their favor.
Polarization of language: Words or phrases can take on drastically different meanings in different political contexts or for different groups.
Dog whistles: Some terms may develop coded political meanings that aren't immediately apparent to all audiences.
Weaponization of ambiguity: The lack of clear, authoritative information about a term's new political meaning can be exploited to spread misinformation or propaganda.

Media manipulators often manipulate language to shape public opinion.
Words are carefully chosen to present a particular viewpoint.
Changes in political ideologies can alter the meaning of terms.
Broader societal shifts influence political discourse.

These elements can rapidly change the connotations and denotations of words, making them challenging to track and analyze.

data void isn't just about a lack of information, but about the potential for confusion, manipulation, or misinformation as terms are redefined or recontextualized in the political sphere.

This type of data void highlights the dynamic nature of language in political discourse and the challenges it presents for maintaining shared understanding in public debates.
It also underscores the importance of being aware of how language can be used as a political tool.



The following papers has been selected to explore this phenomenon.
\begin{itemize}
    \item Statistically Significant Detection of Linguistic Change.
    ACM WWW 2014 ~\cite{kulkarni2014statisticallysignificantdetectionlinguistic}
    \item Diachronic Word Embeddings Reveal Statistical Laws of Semantic Change.
    ACL 2016 ~\cite{hamilton-etal-2016-diachronic}
\end{itemize}

\subsection{What is a Data Void?}\label{subsec:what-is-a-data-void?}
Data voids, as first introduced by Golebiewski and Boyd \cite{unknown},
are identified while searching terms for which relevant information is either limited, non-existent, or deeply problematic.
\\
For example, unlike a typical web search for `basketball' that yields a vast amount of data,
data voids present a problematic scenario such as searches on these terms may return no results or irrelevant, often inaccurate information.
This situation could prompt users to pursue information, click on misleading search outcomes, and be exposed to false or misleading content,
ultimately shaping their perceptions and beliefs with misinformation.

Data Voids often originate in fringe communities.
When vulnerable users search for information using this term, they may encounter results that are misrepresentations of the topic, as these fringe communities have already shaped the narrative.
Mainstream media, particularly those with far-left or far-right biases, may pick up and amplify these narratives, contributing to the misinformation cycle.


%! Author = mkeim
%! Date = 8/13/24

\section{Conclusion}\label{sec:}
In this report, we explored the mechanisms by which words change their meanings over time and how this process can be leveraged to identify data voids.
We examined two key studies to gain insights into this phenomenon.

The first study by Kulkarni et al.\ (2015) presents a robust framework for detecting significant shifts in word usage over time.
Their methodology involves constructing time series for individual words based on frequency, syntactic, and distributional methods, followed by the application of a change point detection algorithm using the Mean Shift model.
This approach effectively identifies moments in time where notable changes in word usage or meaning occur, as illustrated by examples like the evolving meanings of “tape” and “apple,” and the contextual shift of “sandy” post-Hurricane Sandy.

The second study by Hamilton et al.\ (2016) advances the study of semantic change through the development and application of various word embedding models — PPMI, SVD, and SGNS — to track the evolution of word meanings over time.
By aligning embeddings across different periods and employing metrics like Spearman correlation and Procrustes alignment,
they uncover two important statistical laws: the Law of Conformity, which indicates that more frequent words undergo smaller semantic changes,
and the Law of Innovation, which shows that words with higher polysemy are more likely to experience significant shifts in meaning.
Their findings provide valuable tools and insights into the dynamics of language evolution.

Through these discussions, we have gained a deeper understanding of how semantic change can be detected and analyzed,
offering powerful tools for identifying data voids in evolving linguistic landscapes.